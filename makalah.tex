\documentclass[a4paper,12 pt]{article}%jenis kertas latex yang dibikin
\usepackage{graphicx}%biar biso ganti2 warno (dk usah dpp)
\usepackage{color}%biar biso ganti2 warno (dk usah dpp)
\usepackage{amsmath}%biar biso lambang math (dk usah dpp)
\usepackage{amsfonts}%biar biso bikin lambang bilangan real (dk usah dpp)
\usepackage[style=authoryear,sorting=nyt,citestyle=authoryear]{biblatex}%untuk referensi pake file bib
\usepackage{arabtex}%biar biso tulis bhs arab
\usepackage{utf8}%biar biso arab jg (perlu 2)

\bibliography{data}%namo file bib nyo "data"
\DefineBibliographyStrings{english}{ %untuk ngubah dikit ado kato yang dak te-translate
    in  = {di},
    and = {dan},
    pages = {hal.}
}
\renewcommand\nameyeardelim{, }%biar pas nge-cite ado komanyo (Name, year)

\renewcommand{\thesubsection}{\Alph{subsection}.}%biar subsectionnyo A. bukan 0.A
\renewcommand{\thesubsubsection}{\arabic{subsubsection}.}%biar sub sub nyo 1., 2. bae bukan 0.A.1
%\renewcommand{namo bagian yang nak diubah penomoran}{cak mano penomoran nyo}
\usepackage{geometry}
\geometry{
    top = 4 cm,
    left = 4 cm,
    bottom = 3 cm,
    right = 3 cm
}
\usepackage{tocloft}%package utk method dibawah 
\renewcommand{\cftpartleader}{\cftdotfill{\cftdotsep}}%biar bagian yang part dikasih titik2 di Daftar Isi

\usepackage{titletoc}%ngatur jarak antar part di Daftar isi

    \titlecontents{part}%namo jenis judul yang dipilih
    [0em] %indentationnyo di Daftar isi
    {\vspace{8pt}\bfseries} %{jarak vertikal antar parts dgn yg lain|jenis textny nak di bold/italic/dll}
    {\thecontentslabel\quad}%numbered parts
    {}%numberless parts
    {\titlerule*[0.7pc]{.}\contentspage}

    \titlecontents{section}%cak pocok tulah
    [0em] %
    {\vspace{8pt}\bfseries}
    {\thecontentslabel\quad}%numbered section
    {}%numberless section
    {\hfill\contentspage}

\title{\textbf{Makalah}\linebreak \\ 
\textbf{Konsep Islam terhadap Isu-Isu Kontemporer(Kebudayaan Islam)}\\[1\baselineskip] 
\normalsize{\textbf{(Disusun untuk tugas mata kuliah Pendidikan Agama Islam)}}}

\date{}
\author{}

\begin{document}
\maketitle
\pagenumbering{roman}%untuk mulai ngitungi page nyo pake i ii iii
\thispagestyle{empty}%untuk ngilangi halaman (letaknyo mesti dibawah maketitle soalnyo dalam \maketitle lah ado pagestyle internally)

\begin{center}%
\includegraphics[width=5cm]{unsri.png}
\end{center}
\vspace{1 cm}

\addcontentsline{toc}{part}{JUDUL}%buat part hantu, tak tertulis tapi ada

\begin{center}
    Disusun oleh:\\[1\baselineskip]
\begin{tabular}{ll} %bukan satu satu (11) tapi el el (ll)
Nur Su'udiyah &: 09021182126020\\
Muhammad Dary Alfaris &: 09021282126038\\
Rajab Agung &: 09021282126060\\
Muhammad Fadhil Al-Fatih &: 09021282126078\\
M Agil Faturrahman &: 09021282126110\\
Fascal Harya Putra &: 09021282126116
\end{tabular}\\[2\baselineskip]

Prodi Teknik Informatika\\
Fakultas Ilmu Komputer\\ 
\textit{Universitas Sriwijaya}\linebreak 
Palembang 2021\linebreak 
\end{center}
\clearpage%untuk langsung ke page baru

\addcontentsline{toc}{part}{KATA PENGANTAR}
\begin{center}
    \section*{KATA PENGANTAR}
\end{center}
%\setcode{utf8}%perlu ditulis ini men nak arab
%\<
%السَلامُ عَليكم ورَحمةُ الله وبَركاته 
%> %mesti dikasih \<...> utk nulis arab
Atas nikmat dan limpahan karunia Allah SWT, Atas Izin-Nya, kami dapat menyelesaikan makalah ini 
dengan tepat waktu. Tak lupa pula shalawat serta salam kepada junjungan Nabi Besar Muhammad SAW.
Beserta keluarganya, para sahabatnya, dan seluruh ummatnya yang istiqomah hingga akhir zaman.

Penulisan makalah ini bertujuan untuk memenuhi tugas kelompok mata kuliah Pendidikan Agama Islam
berjudul Konsep Islam terhadap Isu-Isu Kontemporer(Kebudayaan Islam).

Dalam makalah ini kami menguraikan materi mengenai bagaimana seharusnya kita menanggapi isu-isu
zaman sekarang dari berbagai sumber, dan dengan opini yang senetral mungkin (tidak berpihak ke mana 
pun).

Kami menyadari bahwa makalah ini masih jauh dari sempurna. Karena itu kami akan terus belajar
dan membuat makalah yang lebih baik lagi di masa mendatang. Harapan kami semoga makalah ini 
akan bermanfaat. 
\\
\begin{flushright}
Palembang, September 2021\\[1\baselineskip]%untuk tambah 1 enter setelah nulis "\\"

Penyusun
\end{flushright}
\clearpage

\thispagestyle{empty} %biar katek nomor jg
\renewcommand{\contentsname}{\hfill\bfseries\Large DAFTAR ISI\hfill}%ngatur letak tempat "DAFTAR ISI"
\renewcommand{\cftaftertoctitle}{\hfill}%ngatur line sesudah judul daftar isi, \hfill itu buat ngisi full kosong
\addcontentsline{toc}{part}{DAFTAR ISI}%biar Daftar isi temasuk kedalem daftar isi (bikin section hantu, tak tertulis tapi ada)
\tableofcontents
\clearpage

\pagenumbering{arabic}%buat mulai nomori pagenyo ke 1
\addcontentsline{toc}{section}{BAB I}%section hantu
\begin{center}
    \section*{BAB I\\PENDAHULUAN}%section tanpa nomor
\end{center}
%\setcounter{section}{1}%Biar ado nomor sectionnyo, soalnyo kito milih section yang katen nomor sblmny
\setcounter{subsection}{0}%Biar nomor subsection mulai dari 1 atau A
\subsection{Latar Bekalang}
Teknologi, keadaan sosial, budaya, dan adab sudah sangat berbeda jika kita bandingkan dengan zaman 
dahulu. Dimulai dari perkembangan teknologi yang berhasil menciptakan alat untuk melakukan
kalkulasi, dapat pula mengeluarkan keluaran untuk ditangkap oleh alat lain, dan juga sistem
penghubung besar yang berjalan dengan baik, membuat perpindahan informasi yang sangat cepat 
bisa terjadi. Keadaan sosial yang dulunya sangat lumrah terjadi perbudakan, tidak memberi hak
kepada wanita, dan membeda-bedakan hak istimewa sesuai dengan ras. Sekarang hak asasi manusia sudah 
berlaku dengan sangat baik, perbudakan sudah menjadi masa lalu kelam bagi umat manusia dan tidak
terjadi lagi, hak-hak wanita sudah ada dan berjalan dengan baik, dan rasisme sudah mulai berkurang.
Bahkan, bukan hanya zaman saja yang membedakan ini, tetapi perbedaan wilayah juga dapat membuat
keadaan keadaan sosial sangat berbeda. 

Oleh karena itu, tentu tantangan zaman sekarang akan jadi sangat berbeda dengan tantangan yang 
dihadapi pada zaman dahulu. Akan ada banyak hal-hal yang bisa diterima jika dilakukan pada zaman
dahulu, tetapi sangat tidak diterima jika dilakukan pada zaman sekarang, begitu juga sebaliknya,
ada banyak hal-hal yang bisa diterima jika dilakukan pada zaman dahulu, tetapi sangat tidak 
diterima pada zaman sekarang. Demikian juga seperti yang disampaikan sebelumnya, perbedaan wilayah juga
dapat menjadi faktor apakah hal-hal tersebut dapat diterima oleh orang-orang atau tidak dapat. 
Akan selalu ada jawaban tepat yang berbeda untuk masa yang berbeda juga, dan tidak akan ada jawaban
yang paling sempurna untuk menjawabnya. Akan selalu ada pertentangan dan kekurangan pada setiap
hal yang dilakukan manusia.

Isu-isu kontemporer berarti isu-isu yang ada pada zaman sekarang, yang pastinya setiap
isu itu sangat kompleks dan tidak mungkin dapat dijelaskan dengan sempurna. Isu-isu kontemporer sangat bervariasi, ada yang hanya terjadi di indonesia, ada juga yang terjadi
di seluruh dunia, atau terjadi hanya di sebagian wilayah lain. Isu-isu kontemporer ada yang
berkaitan dengan ideogi Islam sendiri, dan ada juga yang sama sekali tidak bersangkutan dengan
ideologi Islam. Banyak orang yang memiliki pendapat berbeda-beda dalam menanggapi isu-isu tersebut. 
Konsep Islam terhadap isu-isu kontemporer berarti bagaimana dengan konsep Islam kita dapat 
menanggapi isu-isu pada zaman sekarang ini. Seperti yang sudah disebutkan sebelumnya, banyak 
orang termasuk para ahli memiliki pendapat yang berbeda-beda untuk menjawab isu-isu yang kompleks 
ini. Maka dalam makalah ini akan menjelaskan beberapa isu-isu tersebut dengan sejelas dan 
sesingkat mungkin dan memuat beberapa pendapat yang dapat menjawab isu-isu zaman sekarang 
yang sangat kompleks sesuai dengan konsep Islam yang telah dipelajari.

\subsection{Rumusan Masalah}
Rumusan masalah dalam makalah ini adalah:
\begin{enumerate}
    \item Apa yang dimaksud dengan isu kontemporer?
    \item Bagaimana isu kontemporer dapat terjadi?
    \item Bagaimana sejarah berjalannya isu tersebut?
\end{enumerate}

\subsection{Tujuan Penulisan}
Tujuan Penulisan makalah ini adalah:
\begin{enumerate}
    \item Menjelaskan isu-isu kontemporer.
    \item Menjelaskan alasan terjadinya isu tersebut.
    \item Menjelaskan sejarah isu kontemporer.
\end{enumerate}
\clearpage

%TEMPLATE BIKIN JUDUL BAB_____________________________________________
\addcontentsline{toc}{section}{BAB II}%section hantu
\begin{center}
    \section*{BAB II\\PEMBAHASAN}
\end{center}
%\setcounter{section}{2}%biar sectionny jadi nomor 2
\setcounter{subsection}{0}%biar nomor subsectionnyo balek lagi jadi 1
%_____________________________________________________________________

\subsection{Islam Liberal}
\subsubsection{Pengertian}
Secara bahasa, Islam liberal berarti Islam yang memiliki kebebasan tanpa batas dan bertanggung 
jawab. Menurut Charles Kurzman, di dalam bukunya \textit{Liberal Islam, A Sourcebook}, menyebut enam gagasan 
yang dapat dipakai sebagai tolak ukur sebuah pemikiran Islam dapat disebut "Liberal" yaitu: 
\begin{enumerate}
    \item Melawan teokrasi, yaitu ide-ide yang hendak mendirikan negara Islam; 
    \item Mendukung gagasan demokrasi; 
    \item Membela hak-hak perempuan; 
    \item Membela hak-hak non-Muslim; 
    \item Membela kebebasan berpikir; 
    \item Membela gagasan kemajuan; 
\end{enumerate}
Siapapun saja, menurut Kurzman, yang membela salah satu dari enam gagasan di atas, maka ia adalah 
seorang Islam Liberal\parencites{zaki}.
\subsubsection{Asal-usul}
Pemikiran liberal Islam mempunyai akar yang jauh sampai di masa keemasan Islam (the golden age of 
Islam). Teologi rasional Islam yang dikembangkan oleh Mu'tazilah  dan para filsuf, seperti al-Kindi, 
al-Farabi, Ibn Sina, Ibn Rusyd dan sebagainya. Liberalisme Islam mendapatkan momentum secara 
politis lebih mendalam pada saat kesultanan Ottoman di Turki, yang oleh segelintir cendekiawan di 
Konstantinopel dirasakan sebagai ketinggalan zaman, terlalu kaku, dan terlalu religius. Diantara 
tokoh-tokoh cendekiawan itu adalah Sinasi, Ziya Pasha dan Namik Kemal. Di Mesir, juga ada 
tokoh-tokoh sekaliber di Turki yang liberal, seperti Rifa' Badawi Rafi' al-Tahtawi (1801-1873), 
Khayr al-Din Pasha (1810-18819), dan Butrus al-Bustam (1819-1883)\parencite{zaki}.
\subsubsection{Islam Liberal di Indonesia}
Bibit pemikiran Islam liberal di indonesia sudah muncul semenjak zaman penjajahan, dikarenakan cara 
pemerintahan dan sistem yang dibuat oleh penguasa pada zaman itu. 
Wacana pemikiran Islam liberal muncul kali pertama di Indonesia oleh Greg Barton dalam bukunya 
yang berjudul Gagasan Islam Liberal di Indonesia pada 1999. Semenjak itu, wacana pemikiran Islam 
liberal menjadi popular, yang kemudian diteruskan oleh Charles Kurzman dalam bukunya Liberal Islam 
dan Ulil Abshar Abdalla yang tergabung dalam organisasi JIL. Sebagian tokoh liberal rata-rata 
mulai berkembang pada 1970-an, karena pada waktu itu, kaum intelektual Islam yang mendapatkan 
pengaruh pemikiran dari Timur Tengah dan Barat mulai muncul seperti Harun Nasution dan Abdurrahman 
Wahid\parencite{samsudin}.

\subsection{Islam dan Terorisme}
\subsubsection{Pengertian Terorisme}
Terorisme dapat diartikan sebagai tindakan kekerasan atau ancaman untuk melakukan tindakan 
kekerasan yang ditujukan kepada sasaran acak (tidak ada hubungan langsung dengan pelaku) yang 
berakibat pada kerusakan, kematian, ketakutan, ketidakpastian dan keputusasaan massal. Tindakan 
terorisme tersebut dilakukan dalam rangka memaksakan kehendak kepada pihak yang dianggap lawan 
oleh kelompok teroris, agar kepentingan-kepentingan mereka diakui dan dihargai\parencite{mustofa}.
\subsubsection{Sejarah Perkembangan Terorisme}
Berdasarkan bebrapa literature, bahwa sesungguhnya sejarah terorisme telah ada sejak beberapa 
abad yang lalu, seiring dengan sejarah kehidupan manusia. Lembaran sejarah manusia telah diwarnai 
oleh tindakan-tindakan teror mulai perang psikologis yang ditulis oleh Xenophon (431-350 SM), 
Kaisar Tiberius (14-37 SM) dan Caligula (37-41 SM) dari Romawi telah mempraktekkan terorisme 
dalam penyingkiran atau pembuangan, perampasan harta benda dan menghukum lawan-lawan politiknya. 
Roberspierre (1758- 1794) meneror musuh-musuhnya dalam masa Revolusi Perancis\parencite{junaid}.
\subsubsection{Asal-Usul Terjadinya Terorisme atas Nama Islam}
\textit{``Terrorists use \textit{ijtihad} to emphasize Quranic clauses that sanction the use of 
violent Jihad as a method ordained by God to preserve the \textit{Shariat} in an Islamic community. 
The manner in which terrorists use \textit{ijtihad} to contextualize geopolitical factors as a 
cause for violent Jihad is determined by their extreme interpretations of the Quran. These 
interpretations also determine the extent of violence used in a Jihad for religious amelioration.''}
\parencite{venkat}

Teroris menggunakan intepretasi ekstrim terhadap ayat-ayat Al-Quran tertentu, lalu dijadikan sebagai 
pembenaran penggunaan metode kekerasan ekstrim untuk melanjutkan aturan-aturan komunitasnya.
Penggunaan intepretasi ekstrim ini pun menjadi penyebab banyak kekerasan-kekerasan yang 
mengatasnamakan jihad untuk "kebangkitan agama" menurut mereka. 
\clearpage

\addcontentsline{toc}{section}{BAB III}%section hantu
\begin{center}
    \section*{BAB III\\PENUTUP}
\end{center}
%\setcounter{section}{3}%biar sectionny jadi nomor 2
\setcounter{subsection}{0}%biar nomor subsectionnyo balek lagi jadi 1

\subsection{Kesimpulan}
Berdasarkan pembahasan diatas, dapat kita simpulkan bahwa pemikiran bisa berkembang menjadi jauh
berbeda dari awal pemikiran tersebut dibuat. Seluruh perbuatan manusia memiliki alasan dan tujuannya
masing-masing. Maka dari itu kita perlu berusaha berkepala dingin untuk menghadapi perbedaan 
pendapat maupun pemikiran. Islam merupakan agama rahmatan lil alamin, yaitu agama yang merupakan 
bentuk rahmat dan rasa kasih sayang Allah SWT kepada seluruh alam semesta, Maka tidak ada alasan
yang mengatasnamakan Islam untuk melakukan kerusakan dan menakut-nakutkan orang lain. Sebagai 
seorang muslim kita perlu mengedepankan kedamaian dunia. \textit{Wallahu'alam bishawab}.

\clearpage
\nocite{*}
\addcontentsline{toc}{part}{DAFTAR PUSTAKA}%biar pustaka masuk dalam pustaka (bikin section hantu, tak tertulis tapi ada)
\printbibliography[title=DAFTAR PUSTAKA]{}

\end{document}